%!TEX TS-program = xelatex
%!TEX encoding = UTF-8 Unicode
% Awesome CV LaTeX Template for CV/Resume
%
% This template has been downloaded from:
% https://github.com/posquit0/Awesome-CV
%
% Author:
% Claud D. Park <posquit0.bj@gmail.com>
% http://www.posquit0.com
%
%
% Adapted to be an Rmarkdown template by Mitchell O'Hara-Wild
% 23 November 2018
%
% Template license:
% CC BY-SA 4.0 (https://creativecommons.org/licenses/by-sa/4.0/)
%
%-------------------------------------------------------------------------------
% CONFIGURATIONS
%-------------------------------------------------------------------------------
% A4 paper size by default, use 'letterpaper' for US letter
\documentclass[11pt, a4paper]{awesome-cv}

% Configure page margins with geometry
\geometry{left=1.4cm, top=.8cm, right=1.4cm, bottom=1.8cm, footskip=.5cm}

% Specify the location of the included fonts
\fontdir[fonts/]

% Color for highlights
% Awesome Colors: awesome-emerald, awesome-skyblue, awesome-red, awesome-pink, awesome-orange
%                 awesome-nephritis, awesome-concrete, awesome-darknight

\definecolor{awesome}{HTML}{414141}

% Colors for text
% Uncomment if you would like to specify your own color
% \definecolor{darktext}{HTML}{414141}
% \definecolor{text}{HTML}{333333}
% \definecolor{graytext}{HTML}{5D5D5D}
% \definecolor{lighttext}{HTML}{999999}

% Set false if you don't want to highlight section with awesome color
\setbool{acvSectionColorHighlight}{true}

% If you would like to change the social information separator from a pipe (|) to something else
\renewcommand{\acvHeaderSocialSep}{\quad\textbar\quad}

\def\endfirstpage{\newpage}

%-------------------------------------------------------------------------------
%	PERSONAL INFORMATION
%	Comment any of the lines below if they are not required
%-------------------------------------------------------------------------------
% Available options: circle|rectangle,edge/noedge,left/right

\photo{contactinfo/pic2.jpg}
\name{Dr.}{Paul Schmidt}

\position{Data Scientist / Biostatistiker}
\address{Hamburg, Deutschland}

\mobile{+49 172 3091577}
\email{\href{mailto:schmidtpaul1989@outlook.com}{\nolinkurl{schmidtpaul1989@outlook.com}}}
\researchgate{Paul\_Schmidt17}
\github{SchmidtPaul}
\linkedin{schmidtpaul1989}

% \gitlab{gitlab-id}
% \stackoverflow{SO-id}{SO-name}
% \skype{skype-id}
% \reddit{reddit-id}


\usepackage{booktabs}

\providecommand{\tightlist}{%
	\setlength{\itemsep}{0pt}\setlength{\parskip}{0pt}}

%------------------------------------------------------------------------------



% Pandoc CSL macros
\newlength{\cslhangindent}
\setlength{\cslhangindent}{1.5em}
\newlength{\csllabelwidth}
\setlength{\csllabelwidth}{3em}
\newenvironment{CSLReferences}[3] % #1 hanging-ident, #2 entry spacing
 {% don't indent paragraphs
  \setlength{\parindent}{0pt}
  % turn on hanging indent if param 1 is 1
  \ifodd #1 \everypar{\setlength{\hangindent}{\cslhangindent}}\ignorespaces\fi
  % set entry spacing
  \ifnum #2 > 0
  \setlength{\parskip}{#2\baselineskip}
  \fi
 }%
 {}
\usepackage{calc}
\newcommand{\CSLBlock}[1]{#1\hfill\break}
\newcommand{\CSLLeftMargin}[1]{\parbox[t]{\csllabelwidth}{#1}}
\newcommand{\CSLRightInline}[1]{\parbox[t]{\linewidth - \csllabelwidth}{#1}}
\newcommand{\CSLIndent}[1]{\hspace{\cslhangindent}#1}

\begin{document}

% Print the header with above personal informations
% Give optional argument to change alignment(C: center, L: left, R: right)
\makecvheader

% Print the footer with 3 arguments(<left>, <center>, <right>)
% Leave any of these blank if they are not needed
% 2019-02-14 Chris Umphlett - add flexibility to the document name in footer, rather than have it be static Curriculum Vitae
\makecvfooter
  {März, 2021}
    {Dr. Paul Schmidt~~~·~~~Curriculum Vitae}
  {\thepage}


%-------------------------------------------------------------------------------
%	CV/RESUME CONTENT
%	Each section is imported separately, open each file in turn to modify content
%------------------------------------------------------------------------------



\makecvfooter{März, 2021}{Dr. Paul Schmidt~~~·~~~Lebenslauf}{\thepage/\pageref{LastPage}}

\hypertarget{berufserfahrung}{%
\section{Berufserfahrung}\label{berufserfahrung}}

\begin{cventries}
    \cventry{Data scientist}{BioMath - Applied Statistics and Informatics in Life Sciences}{Rostock \& Hamburg}{Seit Jan 2019}{\begin{cvitems}
\item Verschiedene statistische Analysen von Rohdaten bis zum Schlussbericht für z.B. jährliches post-market Monitoring (Umfrage; Landwirtschaft), Risikobewertung (Metaanalyse; Epidemiologie), mehrjähriger Feldversuche (Experiment; Umwelt), Geografische Verteilung (GIS; Landesamt)
\item Implementierung neuer / Optimierung vorhandener SOPs (z.B. für systematic literature reviews und Metaanalysen), indem beispielsweise die Funktionalität vorhandener Software besser genutzt wird und zusätzlich ergänzende Software/Tools eingesetzt werden
\item Koordination der Kommunikation und des Zeitmanagements von Projekten
\item Durchführung von detaillierten Recherchen und Verfassen von wissenschaftlichen Texten
\item Entwicklung und Durchführung von statistischen Workshops
\item Zuarbeit für / Korrekturlesen von speziell englischen Entwürfen von Anträgen, Berichten und wissenschaftlichen Publikationen
\end{cvitems}}
    \cventry{Workshop Coach}{Freelancer (nebenberuflich)}{siehe `Lehre' Abschnitt unten}{Seit Nov 2018}{\begin{cvitems}
\item Entwicklung und Durchführung von Workshops zu Statistik mit R; der genaue Inhalt und die Kurssprache in Absprache mit dem Auftraggeber
\item Bereitstellung des Kursmaterials auf eigener Webseite (siehe 'Weitere Fähigkeiten' Abschnitt unten)
\end{cvitems}}
    \cventry{Wissenschaftlicher Mitarbeiter}{Universität Hohenheim}{Stuttgart}{Sep 2015 - Dez 2018}{\begin{cvitems}
\item Persönliche Beratung (von Einzeltermin bis projektbegleitend) für Studenten und wissenschaftliche Mitarbeiter hinsichtlich Versuchsdesign, Datenverarbeitung, statistischer Analysen und/oder Ergebnisdarstellung
\item Entwicklung, Organisation und Durchführung jährlicher statistischer Auswertungen von Versuchen zur Ertragsstabilität für eine externe Firma
\item Entwicklung, Organisation und Durchführung von Workshops zu Statistik mit R und SAS
\item Betreuung einer MSc Thesis
\end{cvitems}}
    \cventry{Junior Data scientist}{BioMath - Applied Statistics and Informatics in Life Sciences}{Rostock, Germany}{Jan 2015 - Aug 2015}{\begin{cvitems}
\item Optimierung statistischer Analysen von monitoring-Daten
\item Implementierung von SOPs zu systematic literature reviews
\end{cvitems}}
\end{cventries}

\hypertarget{ausbildung}{%
\section{Ausbildung}\label{ausbildung}}

\begin{cventries}
    \cventry{Dr. sc. agr.}{Universität Hohenheim}{Stuttgart}{Sep 2015 - Nov 2019}{\begin{cvitems}
\item DFG-geförderter Doktorand im Fachgebiet Biostatistik unter Prof. Dr. Hans-Peter Piepho
\item Kumulative Doktorarbeit: 'Estimating heritability in plant breeding programs' benotet mit 'magna cum laude'
\end{cvitems}}
    \cventry{Visiting PhD student}{Purdue University}{West Lafayette, IN, USA}{Sep 2015 - Dez 2015}{\begin{cvitems}
\item Gastdoktorand im Fachgebiet statistical bioinformatics unter Prof. Dr. Rebecca Whitbeck Doerge
\item Durch Eigeninitiative organisiert um den wissenschaftlichen Austausch und so die Inspiration zu Beginn meiner Doktorarbeit anzuregen
\end{cvitems}}
    \cventry{MSc Crop Science: Plant Breeding}{Universität Hohenheim}{Stuttgart}{Okt 2012 - Dez 2014}{\begin{cvitems}
\item Vertiefung in Biostatistik und Pflanzenzüchtung (Gesamtnote 1,4)
\item MSc Thesis: 'Statistical Evaluation and Analysis of PACTS trials as a series of on-farm strip trials without replicates' benotet mit 1,0
\end{cvitems}}
    \cventry{BSc Agrarbiologie}{Universität Hohenheim}{Stuttgart}{Okt 2009 - Sep 2012}{\begin{cvitems}
\item Vertiefung in Genetik und Pflanzenwissenschaften  (Gesamtnote 1,9)
\item BSc Thesis: 'Cumulative effects of glyphosate trace concentrations during root exposition of winter wheat' benotet mit 1,0
\end{cvitems}}
    \cventry{Schüleraustausch}{Alexander Central High School}{Taylorsville, NC, USA}{Aug 2006 - Jul 2007}{\begin{cvitems}
\item Vollendung des Abschlussjahres samt Erhalt eines High School Diploms
\end{cvitems}}
\end{cventries}

\hypertarget{weitere-fuxe4higkeiten}{%
\section{Weitere Fähigkeiten}\label{weitere-fuxe4higkeiten}}

\begin{cvskills}

  \cvskill  {Generell}
  {Teamfähigkeit, Kommunikation, strukturiertes Arbeiten, Zeitmanagement, Problemlösung, zielorientiert} 
  
  \cvskill  {Sprachen}
  {Deutsch (Muttersprache), Englisch (kompetente, professionelle Sprachverwendung)}
  
  \cvskill  {Software}
  {R, SAS, SPSS, ASReml, Excel, Word, PowerPoint, Citavi, Adobe Acrobat Pro, Latex, C\#, SQL} 
  
  \cvskill  {Statistik}
  {(generalisierte) lineare (gemischte) Modelle, explorative \& deskriptive Datenauswertung, Versuchsdesign} 
  
  \cvskill  {Präsentation}
  {Datenvisualisierung, Datenanalysebericht, wissenschaftliche Publikationen, Präsentationen} 
  
  \cvskill{Webseiten}{https://schmidtpaul.github.io/MMFAIR/0contactinfo.html}

\end{cvskills}

\hypertarget{lehre}{%
\section{Lehre}\label{lehre}}

\begin{cvhonors}

  \cvhonor
    {Workshopleiter  }
    {Data Science in den Naturwiss. mit R (Teil 2)}
    {Thünen Inst. Braunschweig (via zoom), 3d}
    {Mär 2021  }  

  \cvhonor
    {Workshopleiter  }
    {Planning exp. designs, rep. measures, and their analyses in R}
    {Uni Kassel (via zoom), 2d}
    {Nov 2020  }
    
  \cvhonor
    {Workshopleiter  }
    {Data Science in den Naturwiss. mit R (Teil 1)}
    {Thünen Inst. Braunschweig (via zoom), 3d}
    {Nov 2020  } 
    
  \cvhonor
    {Workshopleiter  }
    {Experimental Design - Practicals in R}
    {CIHEAM Saragossa (via zoom), 2d}
    {Okt 2020  }

  \cvhonor
    {Workshopleiter  }
    {Real-time consultation on statistics and mixed models in R}
    {Uni Kassel, 2d}
    {Mär 2020  }
    
  \cvhonor
    {Workshopleiter  }
    {Basics of applied statistics}
    {Uni Rostock, 2d}
    {Dez 2019  }
  
  \cvhonor
    {Workshopleiter  }
    {Data Science in den Naturwiss. mit R (Teil 2)}
    {Thünen Inst. Braunschweig, 3d}
    {Nov 2019  }    

  \cvhonor
    {Workshopleiter  }
    {Data Science in den Naturwiss. mit R (Teil 1)}
    {Thünen Inst. Braunschweig, 3d}
    {Okt 2019  }      

  \cvhonor
    {Workshopleiter  }
    {Essential basics of statistics}
    {Uni Rostock, 2d}
    {Sep 2019  }  

  \cvhonor
    {Workshopleiter  }
    {Gemischte Modelle in R}
    {Thünen Inst. Braunschweig, 3d}
    {Nov 2018  }  
    
  \cvhonor
    {Workshopleiter  }
    {Implementation of yield stability assessment with ASReml-R}
    {Bangladesh Rice Research Inst., Gazipur, 3h}
    {Mai 2018  }
    
  \cvhonor
    {Workshopleiter  }
    {Statistical analysis with SAS (monatlich)}
    {Uni Hohenheim, Stuttgart, 3d}
    {2016-2018  }
    
  \cvhonor
    {Workshopleiter  }
    {Statistical analysis with R (monatlich)}
    {Uni Hohenheim, Stuttgart, 3d}
    {2016-2018  }
    
  \cvhonor
    {Lehrassistent  }
    {Biometrie / Statistik (wöchentlich)}
    {Uni Hohenheim, Stuttgart, 4h}
    {2016-2018  }
        
\end{cvhonors}

\hypertarget{wissenschaftliche-publikationen}{%
\section{Wissenschaftliche
Publikationen}\label{wissenschaftliche-publikationen}}

\footnotesize

\hypertarget{bibliography}{}
\leavevmode\hypertarget{ref-Buntaran2020}{}%
\CSLLeftMargin{1. }
\CSLRightInline{Buntaran, H., Piepho, H.-P., Schmidt, P., Ryden, J.,
Halling, M., \& Forkman, J. (2020). Cross-validation of stagewise
mixed-model analysis of swedish variety trials with winter wheat and
spring barley. \emph{Crop Science}, \emph{60}(5), 2221--2240.
\url{https://doi.org/10.1002/csc2.20177}}

\leavevmode\hypertarget{ref-Kukowski2020}{}%
\CSLLeftMargin{2. }
\CSLRightInline{Kukowski, S., Schmidt, P., Piepho, H.-P., Roehl, M.,
Hauffe, H.-K., \& Streck, T. (2020). Auswirkungen atmosphaerischer
stickstoffeintraege auf magere flachland-maehwiesen in
baden-wuerttemberg. \emph{Natur Und Landschaft}, \emph{95}(2), 58--67.
\url{https://doi.org/10.17433/2.2020.50153773.58-67}}

\leavevmode\hypertarget{ref-Schmidt2019}{}%
\CSLLeftMargin{3. }
\CSLRightInline{Schmidt, P. (2019). Estimating heritability in plant
breeding programs. In \emph{Dissertation to obtain the doctoral degree
of Agricultural Sciences (Dr.~sc. agr.): Vol. -} (Issue -, p. --).
\url{https://doi.org/-}}

\leavevmode\hypertarget{ref-Schmidt2019b}{}%
\CSLLeftMargin{4. }
\CSLRightInline{Schmidt, P., Hartung, J., Bennewitz, J., \& Piepho,
H.-P. (2019). Heritability in plant breeding on a genotype-difference
basis. \emph{Genetics}, \emph{212}(4), 991--1008.
\url{https://doi.org/10.1534/genetics.119.302134}}

\leavevmode\hypertarget{ref-Schmidt2019c}{}%
\CSLLeftMargin{5. }
\CSLRightInline{Schmidt, P., Hartung, J., Rath, J., \& Piepho, H.-P.
(2019). Estimating broad-sense heritability with unbalanced data from
agricultural cultivar trials. \emph{Crop Science}, \emph{59}(2),
525--536. \url{https://doi.org/10.2135/cropsci2018.06.0376}}

\leavevmode\hypertarget{ref-Schmidt2018}{}%
\CSLLeftMargin{6. }
\CSLRightInline{Schmidt, P., Moehring, J., Koch, R. J., \& Piepho, H.-P.
(2018). More, larger, simpler: How comparable are on-farm and on-station
trials for cultivar evaluation? \emph{Crop Science}, \emph{58}(4),
1508--1518. \url{https://doi.org/10.2135/cropsci2017.09.0555}}

\leavevmode\hypertarget{ref-Tulinska2018}{}%
\CSLLeftMargin{7. }
\CSLRightInline{Tulinska, J., Adel-Patient, K., Bernard, H., Liskova,
A., Kuricova, M., Ilavska, S., Horvathova, M., Kebis, A., Rollerova, E.,
Babincova, J., Alacova, R., Wal, J.-M., Schmidt, K., Schmidtke, J.,
Schmidt, P., Kohl, C., Wilhelm, R., Schiemann, J., \& Steinberg, P.
(2018). Humoral and cellular immune response in wistar han RCC rats fed
two genetically modified maize MON810 varieties for 90 days (EU 7th
framework programme project GRACE). \emph{Archives of Toxicology},
\emph{92}(7), 2385--2399.
\url{https://doi.org/10.1007/s00204-018-2230-z}}

\leavevmode\hypertarget{ref-Schmidt2017}{}%
\CSLLeftMargin{8. }
\CSLRightInline{Schmidt, K., Schmidtke, J., Schmidt, P., Kohl, C.,
Wilhelm, R., Schiemann, J., van der Voet, H., \& Steinberg, P. (2017).
Variability of control data and relevance of observed group differences
in five oral toxicity studies with genetically modified maize MON810 in
rats. \emph{Archives of Toxicology}, \emph{91}(4), 1977--2006.
\url{https://doi.org/10.1007/s00204-016-1857-x}}

\leavevmode\hypertarget{ref-Zeljenkova2016}{}%
\CSLLeftMargin{9. }
\CSLRightInline{Zeljenkova, D., Alacova, R., Ondrejkova, J., Ambrusova,
K., Bartusova, M., Kebis, A., Kovriznych, J., Rollerova, E., Szabova,
E., Wimmerova, S., Cernak, M., Krivosikova, Z., Kuricova, M., Liskova,
A., Spustova, V., Tulinska, J., Levkut, M., Revajova, V., Sevcikova, Z.,
\ldots{} Steinberg, P. (2016). One-year oral toxicity study on a
genetically modified maize MON810 variety in wistar han RCC rats (EU 7th
framework programme project GRACE). \emph{Archives of Toxicology},
\emph{90}(10), 2531--2562.
\url{https://doi.org/10.1007/s00204-016-1798-4}}

\end{document}
