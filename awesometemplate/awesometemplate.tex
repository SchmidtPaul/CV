%!TEX TS-program = xelatex
%!TEX encoding = UTF-8 Unicode
% Awesome CV LaTeX Template for CV/Resume
%
% This template has been downloaded from:
% https://github.com/posquit0/Awesome-CV
%
% Author:
% Claud D. Park <posquit0.bj@gmail.com>
% http://www.posquit0.com
%
%
% Adapted to be an Rmarkdown template by Mitchell O'Hara-Wild
% 23 November 2018
%
% Template license:
% CC BY-SA 4.0 (https://creativecommons.org/licenses/by-sa/4.0/)
%
%-------------------------------------------------------------------------------
% CONFIGURATIONS
%-------------------------------------------------------------------------------
% A4 paper size by default, use 'letterpaper' for US letter
\documentclass[11pt, a4paper]{awesome-cv}

% Configure page margins with geometry
\geometry{left=1.4cm, top=.8cm, right=1.4cm, bottom=1.8cm, footskip=.5cm}

% Specify the location of the included fonts
\fontdir[fonts/]

% Color for highlights
% Awesome Colors: awesome-emerald, awesome-skyblue, awesome-red, awesome-pink, awesome-orange
%                 awesome-nephritis, awesome-concrete, awesome-darknight

\colorlet{awesome}{awesome-red}

% Colors for text
% Uncomment if you would like to specify your own color
% \definecolor{darktext}{HTML}{414141}
% \definecolor{text}{HTML}{333333}
% \definecolor{graytext}{HTML}{5D5D5D}
% \definecolor{lighttext}{HTML}{999999}

% Set false if you don't want to highlight section with awesome color
\setbool{acvSectionColorHighlight}{true}

% If you would like to change the social information separator from a pipe (|) to something else
\renewcommand{\acvHeaderSocialSep}{\quad\textbar\quad}

\def\endfirstpage{\newpage}

%-------------------------------------------------------------------------------
%	PERSONAL INFORMATION
%	Comment any of the lines below if they are not required
%-------------------------------------------------------------------------------
% Available options: circle|rectangle,edge/noedge,left/right

\name{Marie}{Curie}

\position{Professor}
\address{School of Physics \& Chemistry, École Normale Supérieure}

\mobile{+1 22 3333 4444}
\email{\href{mailto:Marie.Curie@ens.fr}{\nolinkurl{Marie.Curie@ens.fr}}}
\homepage{mariecurie.com}
\github{mariecurie}
\linkedin{mariecurie}
\twitter{mariecurie}

% \gitlab{gitlab-id}
% \stackoverflow{SO-id}{SO-name}
% \skype{skype-id}
% \reddit{reddit-id}


\usepackage{booktabs}

% Templates for detailed entries
% Arguments: what when with where why
\usepackage{etoolbox}
\def\detaileditem#1#2#3#4#5{%
\cventry{#1}{#3}{#4}{#2}{\ifx#5\empty\else{\begin{cvitems}#5\end{cvitems}}\fi}\ifx#5\empty{\vspace{-4.0mm}}\else\fi}
\def\detailedsection#1{\begin{cventries}#1\end{cventries}}

% Templates for brief entries
% Arguments: what when with
\def\briefitem#1#2#3{\cvhonor{}{#1}{#3}{#2}}
\def\briefsection#1{\begin{cvhonors}#1\end{cvhonors}}

\providecommand{\tightlist}{%
	\setlength{\itemsep}{0pt}\setlength{\parskip}{0pt}}

%------------------------------------------------------------------------------


%%%% BIBLIOGRAPHY
% Bibliography formatting

\usepackage[sorting=ynt,citestyle=authoryear,bibstyle=authoryear-comp,defernumbers=true,maxnames=20,giveninits=true, bibencoding=utf8, terseinits=true, uniquename=init,dashed=false,doi=false,isbn=false,natbib=true,backend=biber]{biblatex}

\DeclareFieldFormat{url}{\url{#1}}
\DeclareFieldFormat[article]{pages}{#1}
\DeclareFieldFormat[inproceedings]{pages}{\lowercase{pp.}#1}
\DeclareFieldFormat[incollection]{pages}{\lowercase{pp.}#1}
\DeclareFieldFormat[article]{volume}{\mkbibbold{#1}}
\DeclareFieldFormat[article]{number}{\mkbibparens{#1}}
\DeclareFieldFormat[article]{title}{\MakeCapital{#1}}
\DeclareFieldFormat[article]{url}{}
\DeclareFieldFormat[inproceedings]{title}{#1}
\DeclareFieldFormat{shorthandwidth}{#1}
\DeclareFieldFormat{extradate}{}

% No dot before number of articles
\usepackage{xpatch}
\xpatchbibmacro{volume+number+eid}{\setunit*{\adddot}}{}{}{}

% Remove In: for an article.
\renewbibmacro{in:}{%
  \ifentrytype{article}{}{%
  \printtext{\bibstring{in}\intitlepunct}}}

%\makeatletter
%\DeclareDelimFormat[cbx@textcite]{nameyeardelim}{\addspace}
%\makeatother

\setlength{\bibitemsep}{1.8pt}
\setlength{\bibhang}{.9cm}
%\renewcommand{\bibfont}{\fontsize{12}{14}}

\renewcommand*{\bibitem}{\addtocounter{papers}{1}\item \mbox{}\hskip-0.9cm\hbox to 0.9cm{\hfill\arabic{papers}.~\,}}
\defbibenvironment{bibliography}
{\list{}
  {\setlength{\leftmargin}{\bibhang}%
   \setlength{\itemsep}{\bibitemsep}%
   \setlength{\parsep}{\bibparsep}}}
{\endlist}
{\bibitem}

\renewcommand{\bibfont}{\normalfont\fontsize{10}{12.4}\selectfont}
% Counters for keeping track of papers
\newcounter{papers}

\DeclareSortingTemplate{ty}{
  \sort{
    \field{title}
  }
  \sort{
    \field{year}
  }
}
\DeclareBibliographyCategory{bib-D:/Coding/CV/awesometemplate/curie.bib-4595288}
\bibliography{D:/Coding/CV/awesometemplate/curie.bib}

\begin{document}

% Print the header with above personal informations
% Give optional argument to change alignment(C: center, L: left, R: right)
\makecvheader

% Print the footer with 3 arguments(<left>, <center>, <right>)
% Leave any of these blank if they are not needed
% 2019-02-14 Chris Umphlett - add flexibility to the document name in footer, rather than have it be static Curriculum Vitae
\makecvfooter
  {August 2020}
    {Marie Curie~~~·~~~Curriculum Vitae}
  {\thepage}


%-------------------------------------------------------------------------------
%	CV/RESUME CONTENT
%	Each section is imported separately, open each file in turn to modify content
%------------------------------------------------------------------------------



\hypertarget{some-stuff-about-me}{%
\section{Some stuff about me}\label{some-stuff-about-me}}

\begin{itemize}
\tightlist
\item
  I poisoned myself doing research.
\item
  I was the first woman to win a Nobel prize
\item
  I was the first person and only woman to win a Nobel prize in two different sciences.
\end{itemize}

\hypertarget{education}{%
\section{Education}\label{education}}

\detailedsection{\detaileditem{Informal studies}{1889-91}{Flying University}{Warsaw, Poland}{\empty}\detaileditem{Master of Physics}{1893}{Sorbonne Université}{Paris, France}{\empty}\detaileditem{Master of Mathematics}{1894}{Sorbonne Université}{Paris, France}{\empty}}

\hypertarget{nobel-prizes}{%
\section{Nobel Prizes}\label{nobel-prizes}}

\briefsection{\briefitem{Nobel Prize in Physics}{1903}{Awarded for her work on radioactivity with Pierre Curie and Henri Becquerel}\briefitem{Nobel Prize in Chemistry}{1911}{Awarded for the discovery of radium and polonium}}

\hypertarget{publications}{%
\section{Publications}\label{publications}}

\defbibheading{bib-D:/Coding/CV/awesometemplate/curie.bib-4595288}{}
\addtocategory{bib-D:/Coding/CV/awesometemplate/curie.bib-4595288}{1,
2,
3,
4,
5,
6,
7,
8,
9,
10,
11,
12,
13,
15}
\newrefcontext[sorting=none]\setcounter{papers}{0}\pagebreak[3]
\printbibliography[category=bib-D:/Coding/CV/awesometemplate/curie.bib-4595288,heading=none]\setcounter{papers}{0}

\nocite{1,
2,
3,
4,
5,
6,
7,
8,
9,
10,
11,
12,
13,
15}

\end{document}
